\documentclass[10pt, letterpaper]{article}
\usepackage[cm]{fullpage}
\usepackage{algpseudocode}
\usepackage{algorithm}
\usepackage{graphicx}
\usepackage[section]{placeins}
\usepackage[table]{xcolor}
\usepackage{amsmath}
\usepackage[margin=0.7in]{geometry}

\algrenewcommand\Return{\State \algorithmicreturn{} }%

\title{0-1 Knapsack}
\author{Daiwei Chen \and Joseph Watts}

\begin{document}
\maketitle
	\begin{abstract}
		0-1 Knapsack is the coined phrase which is used to simply describe the problem of finding the maximum value possible by choosing what items to take, when not everything can be taken.
		Put slightly more formally, given a set of items with all items having a value and weight assigned to them and a bag which allows the user to carry unlimited volume but only a limited weight, the goal is to find the set of items which allow the user to get the maximum value from the chosen items.
		This paper explores different approaches to solving this problem, including both a Dynamic Programming approach along with multiple different Greedy approaches.
	\end{abstract}

\section{Background and Related Work}
The 0-1 Knapsack problem is commonly found in project management and economics in dealing with resource allocation. In an oversimplified manner, how can resources divided so that different departments can get the most valuable work done given their needed resources.
0-1 Knapsack also shows up when working with networking equiptment which try to most efficiently distribute load between different stations.
This is known as load balancing and is used in many common day-to-day technologies.
0-1 Knapsack was also used as a key part of the design of public-private keys when first made during the 1970's. A private key is generated with an easy knapsack problem, and then a hard knapsack is derrived by it and this is what is used as the public key.
\section{Greedy Algorithm}

\section{Dynamic Algorithm}

\section{Experimental Setup}

\section{Results}
	% Diagram showing the average time between different approaches
	\begin{figure}[htbp]
		\begin{center}
			\includegraphics[width=0.70\textwidth]{python/accuracyGraph.png}
			\caption{Algorithm Accuracy for Taking Items Given Maximum Weight $n$}
			\label{fig:accuracy-graph}
		\end{center}
	\end{figure}
	% Diagram showing the % accuracy of different approaches
	\begin{figure}[htbp]
		\begin{center}
			\includegraphics[width=0.70\textwidth]{python/timeGraph.png}
			\caption{Time to Calculate Items to take with Maximum Available Weight $n$}
			\label{fig:time-graph}
		\end{center}
	\end{figure}
\section{Conclusions}

\end{document}
